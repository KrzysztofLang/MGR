%%%%%%%%%%%%%%%%%%%%%%%%%%%%%%%%%%%%%%%%%%%%%%%%%%%%%%%%%%%%%%%%%
%%% %
%%% % weiiszablon.tex
%%% % The Faculty of Electrical and Computer Engineering
%%% % Rzeszow University Of Technology diploma thesis Template
%%% % Szablon pracy dyplomowej Wydziału Elektrotechniki 
%%% % i Informatyki PRz
%%% % June, 2015
%%%%%%%%%%%%%%%%%%%%%%%%%%%%%%%%%%%%%%%%%%%%%%%%%%%%%%%%%%%%%%%%%

\documentclass[12pt,twoside]{article}

\usepackage{weiiszablon}

\author{Krzysztof Lang}

% np. EF-123456, EN-654321, ...
\studentID{EF-148853}

\title{Implementacja wybranych algorytmów wypełniania brakujących wartości, dla strumieni dużych zbiorów danych}
\titleEN{Implemetation of selected missing value filling algorithms for large data sets}


%%% wybierz rodzaj pracy wpisując jeden z poniższych numerów: ...
% 1 = inżynierska	% BSc
% 2 = magisterska	% MSc
% 3 = doktorska		% PhD
%%% na miejsce zera w linijce poniżej
\newcommand{\rodzajPracyNo}{2}


%%% promotor
\supervisor{dr Michał Piętal}
%% przykład: dr hab. inż. Józef Nowak, prof. PRz

%%% promotor ze stopniami naukowymi po angielsku
\supervisorEN{Michał Piętal, PhD}

\abstract{Treść streszczenia po polsku}
\abstractEN{Treść streszczenia po angielsku}

\begin{document}

% strona tytułowa
\maketitle

\blankpage

% spis treści
\tableofcontents
\clearpage
\blankpage
\section{Wstęp}
\clearpage
\section{Wprowadzenie do wypełniania brakujących wartości}
\subsection{Rys historyczny}
\subsection{Na czym polega wypełnianie brakujących wartości}
\subsection{Korzyści i zagrożenia}
\subsection{Perspektywy na przyszłość}
\clearpage
\section{Omówienie narzędzi i danych}
\subsection{Biblioteki}
\subsubsection{Pandas}
\subsubsection{NumPy}
\subsubsection{Scikit}
\subsection{Źródła danych}
\subsubsection{Użyte repozytopria danych}
\subsubsection{Źródło 1}
\subsubsection{Źródło 2}
\subsubsection{Źródło 3}
\clearpage
\section{Implementacja i testy algorytmu}
\subsection{Opis implementacji}
\subsubsection{Alg 1}
\subsubsection{Alg 2}
\subsubsection{Alg 3}
\subsection{Napotkane problemy}
\subsection{Testy algorytmów na wybranych źródłach danych}
\clearpage

\section{Podsumowanie i wnioski końcowe}
\clearpage

\section*{Załączniki}
\addcontentsline{toc}{section}{Załączniki}

\clearpage

\addcontentsline{toc}{section}{Literatura}

\begin{thebibliography}{4}
\end{thebibliography}

\clearpage

\makesummary

\end{document} 
